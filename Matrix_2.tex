\documentclass[journal,12pt,twocolumn]{IEEEtran}
%
\usepackage{setspace}
\usepackage{gensymb}
%\doublespacing
\singlespacing

\usepackage[cmex10]{amsmath}
\usepackage{amsthm}
%\usepackage{iithtlc}
\usepackage{mathrsfs}
\usepackage{txfonts}
\usepackage{stfloats}
\usepackage{bm}
\usepackage{cite}
\usepackage{cases}
\usepackage{subfig}
%\usepackage{xtab}
\usepackage{longtable}
\usepackage{multirow}

\usepackage{enumitem}
\usepackage{mathtools}
\usepackage{steinmetz}
\usepackage{tikz}
\usepackage{circuitikz}
\usepackage{verbatim}
\usepackage{tfrupee}
\usepackage[breaklinks=true]{hyperref}
\usepackage{tkz-euclide} % loads  TikZ and tkz-base
\usetikzlibrary{calc,math}
\usepackage{listings}
    \usepackage{color}                                            %%
    \usepackage{array}                                            %%
    \usepackage{longtable}                                        %%
    \usepackage{calc}                                             %%
    \usepackage{multirow}                                         %%
    \usepackage{hhline}                                           %%
    \usepackage{ifthen}                                           %%
  %optionally (for landscape tables embedded in another document): %%
    \usepackage{lscape}     
\usepackage{multicol}
\usepackage{chngcntr}
%\usepackage{enumerate}

%\usepackage{wasysym}
%\newcounter{MYtempeqncnt}
\DeclareMathOperator*{\Res}{Res}
%\renewcommand{\baselinestretch}{2}
\renewcommand\thesection{\arabic{section}}
\renewcommand\thesubsection{\thesection.\arabic{subsection}}
\renewcommand\thesubsubsection{\thesubsection.\arabic{subsubsection}}

\renewcommand\thesectiondis{\arabic{section}}
\renewcommand\thesubsectiondis{\thesectiondis.\arabic{subsection}}
\renewcommand\thesubsubsectiondis{\thesubsectiondis.\arabic{subsubsection}}

% correct bad hyphenation here
\hyphenation{op-tical net-works semi-conduc-tor}
\def\inputGnumericTable{}                                 %%

\lstset{
%language=C,
frame=single, 
breaklines=true,
columns=fullflexible
}
%\lstset{
%language=tex,
%frame=single, 
%breaklines=true
%}

\begin{document}
%


\newtheorem{theorem}{Theorem}[section]
\newtheorem{problem}{Problem}
\newtheorem{proposition}{Proposition}[section]
\newtheorem{lemma}{Lemma}[section]
\newtheorem{corollary}[theorem]{Corollary}
\newtheorem{example}{Example}[section]
\newtheorem{definition}[problem]{Definition}
\newcommand{\BEQA}{\begin{eqnarray}}
\newcommand{\EEQA}{\end{eqnarray}}
\newcommand{\define}{\stackrel{\triangle}{=}}
\bibliographystyle{IEEEtran}
\providecommand{\mbf}{\mathbf}
\providecommand{\pr}[1]{\ensuremath{\Pr\left(#1\right)}}
\providecommand{\qfunc}[1]{\ensuremath{Q\left(#1\right)}}
\providecommand{\sbrak}[1]{\ensuremath{{}\left[#1\right]}}
\providecommand{\lsbrak}[1]{\ensuremath{{}\left[#1\right.}}
\providecommand{\rsbrak}[1]{\ensuremath{{}\left.#1\right]}}
\providecommand{\brak}[1]{\ensuremath{\left(#1\right)}}
\providecommand{\lbrak}[1]{\ensuremath{\left(#1\right.}}
\providecommand{\rbrak}[1]{\ensuremath{\left.#1\right)}}
\providecommand{\cbrak}[1]{\ensuremath{\left\{#1\right\}}}
\providecommand{\lcbrak}[1]{\ensuremath{\left\{#1\right.}}
\providecommand{\rcbrak}[1]{\ensuremath{\left.#1\right\}}}
\theoremstyle{remark}
\newtheorem{rem}{Remark}
\newcommand{\sgn}{\mathop{\mathrm{sgn}}}
\providecommand{\abs}[1]{\left\vert#1\right\vert}
\providecommand{\res}[1]{\Res\displaylimits_{#1}} 
\providecommand{\norm}[1]{\left\lVert#1\right\rVert}
\providecommand{\norm}[1]{\lVert#1\rVert}
\providecommand{\mtx}[1]{\mathbf{#1}}
\providecommand{\mean}[1]{E\left[ #1 \right]}
\providecommand{\fourier}{\overset{\mathcal{F}}{ \rightleftharpoons}}
%\providecommand{\hilbert}{\overset{\mathcal{H}}{ \rightleftharpoons}}
\providecommand{\system}{\overset{\mathcal{H}}{ \longleftrightarrow}}
	%\newcommand{\solution}[2]{\textbf{Solution:}{#1}}
\newcommand{\solution}{\noindent \textbf{Solution: }}
\newcommand{\cosec}{\,\text{cosec}\,}
\providecommand{\dec}[2]{\ensuremath{\overset{#1}{\underset{#2}{\gtrless}}}}
\newcommand{\myvec}[1]{\ensuremath{\begin{pmatrix}#1\end{pmatrix}}}
\newcommand{\mydet}[1]{\ensuremath{\begin{vmatrix}#1\end{vmatrix}}}
%\numberwithin{equation}{section}
\numberwithin{equation}{subsection}
\makeatletter
\@addtoreset{figure}{problem}
\makeatother
\let\StandardTheFigure\thefigure
\let\vec\mathbf
%\renewcommand{\thefigure}{\theproblem.\arabic{figure}}
\renewcommand{\thefigure}{\theproblem}
%\setlist[enumerate,1]{before=\renewcommand\theequation{\theenumi.\arabic{equation}}
%\counterwithin{equation}{enumi}
%\renewcommand{\theequation}{\arabic{subsection}.\arabic{equation}}
\def\putbox#1#2#3{\makebox[0in][l]{\makebox[#1][l]{}\raisebox{\baselineskip}[0in][0in]{\raisebox{#2}[0in][0in]{#3}}}}
     \def\rightbox#1{\makebox[0in][r]{#1}}
     \def\centbox#1{\makebox[0in]{#1}}
     \def\topbox#1{\raisebox{-\baselineskip}[0in][0in]{#1}}
     \def\midbox#1{\raisebox{-0.5\baselineskip}[0in][0in]{#1}}
\vspace{3cm}
\title{Assignment 3}
\author{Rajasekhar Jala}
\maketitle
\newpage
%\tableofcontents
\bigskip
\renewcommand{\thefigure}{\theenumi}
\renewcommand{\thetable}{\theenumi}
Find Python Codes from below link 
%
\begin{lstlisting}
https://github.com/Sekharjala/Matrix-Matrix_2
\end{lstlisting}
%
and latex-tikz codes from 
%
\begin{lstlisting}
https://github.com/Sekharjala/Matrix-Matrix_2
\end{lstlisting}
%
\section{Examples I}
\subsection{Question 14}
Prove that the point \myvec{-1/14 \\ 39/14} is the centre of the circle circumscribing the triangle whose angular points are 
\begin{enumerate}
\begin{align}
\vec{A}&=\myvec{1\\1} & \vec{B}&=\myvec{2\\3} &\vec{C}&=\myvec{-2\\2}
\end{align}
\end{enumerate}
\section{Solution}
\item let assume that circumecentre of the triangle ABC is $\vec{O}$
\begin{align}
\norm{\vec{A}-\vec{O}} = \norm{\vec{B}-\vec{O}} = \norm{\vec{C}-\vec{O}}
\\
\norm{\vec{A}-\vec{O}}^2 - \norm{\vec{B}-\vec{O}}^2 = 0
\end{align}
Which can be simplified as

\begin{align}
\myvec{\vec{A}-\vec{B}}^T \vec{O} = \frac{(\norm{\vec{A}}^2 -\norm{\vec {B}}^2)}{2}
\end{align}
Similarly,
\begin{align}
\myvec{\vec{B}-\vec{C}}^T \vec{O} = \frac{(\norm{\vec{B}}^2 -\norm{\vec{C}}^2)}{2}
\end{align}
can be combined to form the matrix equation 

\begin{align}
\vec{N}^T \vec{O} &= \vec{K} \\
\vec{O} &= \vec{K} {\vec{N}^{-T}} 
\end{align}
Where
\begin{enumerate}
\begin{align}
\vec{N} &= \myvec{(\vec{A} -\vec{B}) & (\vec{B} -\vec{C})} \label{eqref_N} \\
\vec{K} &= \frac{1}{2}\myvec{\norm{\vec{A}}^2 -\norm{\vec{B}}^2 & \norm{\vec{B}}^2 - \norm{\vec{C}}^2} \label{eqref_c}
\end{align}
by substituting $ \vec{A} ,\vec{B} $ and $\vec{C} $ in \eqref{eqref_N} and \eqref{eqref_c} \\
we get
\begin{align}
\vec{N} &= \myvec{-1&4 \\ -2&1}  \\
\vec{N}^{-T}&=\myvec{1/7 &-4/7 \\2/7 &-1/7} \label{eqA} \\ 
\vec{K} &=\frac{1}{2}\myvec{(2 -13)  & (13-8)} \nonumber  \\
\vec{K} &= \myvec{-11/2 & 5/2} \label{eqB} 
\end{enumerate}
\end{align}
from \eqref{eqA} and \eqref{eqB} we get
\begin{align}
\vec{O} &= \myvec{-1/14 & 39/14} 
\end{align}
 Hence in vector form 
\begin{align}
\vec{O} &= \myvec{-1/14 \\  39/14}
\end{align}
\begin{figure}[!h]
\includegraphics[width=\columnwidth]{Matrix_2.png}
\caption{Circumcircle with $\vec{O}$ as center}
\label{fig:}
\end{figure}
\end{enumerate}
\end{document}
